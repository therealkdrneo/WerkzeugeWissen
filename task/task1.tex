\documentclass[a4paper, 12pt]{article}

\usepackage[utf8]{inputenc}
\usepackage[T1]{fontenc}
\usepackage[ngerman]{babel}
\usepackage{graphicx}
\usepackage{hyperref}
\usepackage{enumitem}
\usepackage{titlesec}
\usepackage{svg}

% Einstellungen für Abschnitte
\titleformat{\section}{\bfseries\large}{\thesection.}{1em}{}
\titleformat{\subsection}{\bfseries\normalsize}{\thesubsection.}{1em}{}

\begin{document}

\begin{titlepage}
    \centering
    {\LARGE \textbf{Werkzeuge für das wissenschaftliche Arbeiten}}\\[1.5em]
    {\Large Python for Machine Learning and Data Science}\\[3em]
    \textbf{Abgabe: 15.12.2023}\\[2em]
    \rule{\textwidth}{0.4pt}\\[0.2cm]
    \vfill
\end{titlepage}

\tableofcontents
\newpage

\section{Projektaufgabe}
In dieser Aufgabe beschäftigen wir uns mit Objektorientierung in Python. Der Fokus liegt auf der Implementierung einer Klasse, wobei insbesondere Magic Methods genutzt werden.

\begin{figure}[h!]
    \centering
    \includesvg[width=12cm]{./../diagram/classes_files.svg}
    \caption{Darstellung der Klassenbeziehungen.}
    \label{fig:classes}
\end{figure}

\subsection{Einleitung}
Ein Datensatz besteht aus mehreren Daten, ein einzelnes Datum wird durch ein Objekt der Klasse \texttt{DataSetItem} repräsentiert. Jedes Datum hat einen Namen (Zeichenkette), eine ID (Zahl) und beliebigen Inhalt.

Mehrere Daten, Objekte vom Typ \texttt{DataSetItem}, sollen in einem Datensatz zusammengefasst werden. Die Schnittstelle und benötigten Operationen sind bereits definiert. Die Klasse \texttt{DataSetInterface} gibt die Schnittstelle vor, während die Implementierung der Datensatz-Klasse noch aussteht.

Implementieren Sie eine Klasse \texttt{DataSet} als Unterklasse von \texttt{DataSetInterface}.

\subsection{Aufbau}
Die Aufgabe umfasst drei Dateien:
\begin{itemize}[noitemsep]
    \item \texttt{dataset.py}: Beinhaltet die Klassen \texttt{DataSetInterface} und \texttt{DataSetItem}.
    \item \texttt{implementation.py}: Hier wird die Klasse \texttt{DataSet} implementiert.
    \item \texttt{main.py}: Testet die Klassen und die Schnittstelle \texttt{DataSetInterface}.
\end{itemize}

\subsection{Methoden}
Die Klasse \texttt{DataSet} muss folgende Methoden implementieren (Details in \texttt{dataset.py}):
\begin{itemize}[label=$\bullet$, leftmargin=2em]
    \item \texttt{\_\_setitem\_\_(self, name, id\_content)}: Hinzufügen eines Datums mit Name, ID und Inhalt.
    \item \texttt{\_\_iadd\_\_(self, item)}: Hinzufügen eines \texttt{DataSetItem}.
    \item \texttt{\_\_delitem\_\_(self, name)}: Löschen eines Datums basierend auf seinem Namen.
    \item \texttt{\_\_contains\_\_(self, name)}: Überprüfen, ob ein Datum mit diesem Namen existiert.
    \item \texttt{\_\_getitem\_\_(self, name)}: Abrufen eines Datums über seinen Namen.
    \item \texttt{\_\_and\_\_(self, dataset)}: Schnittmenge zweier Datensätze als neuen Datensatz zurückgeben.
    \item \texttt{\_\_or\_\_(self, dataset)}: Vereinigung zweier Datensätze als neuen Datensatz zurückgeben.
    \item \texttt{\_\_iter\_\_(self)}: Iteration über alle Daten im Datensatz (optional sortiert).
    \item \texttt{filtered\_iterate(self, filter)}: Gefilterte Iteration über den Datensatz mit einer Lambda-Funktion.
    \item \texttt{\_\_len\_\_(self)}: Anzahl der Daten im Datensatz abrufen.
\end{itemize}

\section{Abgabe}
Implementieren Sie die Klasse \texttt{DataSet} in der Datei \texttt{implementation.py} zur Lösung der oben beschriebenen Aufgabe.

Die Datei \texttt{main.py} wird im VPL (Virtual Programming Lab) mit zusätzlichen Tests erweitert, um Ihre Lösung zu überprüfen.

\hrule
\footnotesize Dateien befinden sich im Ordner \texttt{/code/} dieses Git-Repositories.

\end{document}